\section{Kvinta}
  \subsection{Český jazyk - prof. Martina Literáková}
    \hlod{prof: "Které z Desatera přikázání se už nyní hodně porušuje?"\newline
Tomáš B: "Nesesmilníš!"\newline
...\newline
(Rozbor básně. Došli jsme k jistému tématu)\newline
prof: "No a co se stalo pak...?"\newline
Tomáš B: "Já myslím, že sesmilnili."\newline
}
    \hlod{(Prof. shrnuje první pololetí)\newline
prof: "... Štěpáne, myslím si, že ses z té důtky ponaučil, jsem za to velice ráda..."\newline
Štěpán B.: "No, mě jenom překvapuje, že ačkoli mě pan ředitel označil za zloděje, tak jsou mi stále svěřovány až desetitisícové částky." (Štěpán vybíral peníze na lyžák).\newline
}
    \hlod{prof: "Kdy můžeme někomu skočit do řeči a není to neslušné?"\newline
Dandi: "No, třeba když hoří."\newline
}
  \subsection{Matematika - prof. Pavel Boucník}
    \hlod{prof: (mluví o všeobecných třídách) "No, pro ty normální je to ...tím  teda nechci říct, že vy nejste normální... No, vlastně (podívá se nás) ... Prostě, pro ty normální by to bylo takhle!"\newline
}
    \hlod{prof: "Kdo to chápe?" \newline
třída: (přihlásí se asi 5 lidí)\newline
prof: "Výborně, skoro všichni!"\newline
}
    \hlod{prof: "Tak, kdo to má správně?"\newline
třída: (přihlásí se 7 lidí)\newline
prof: "Výborně, to jsou dvě třetiny!"\newline
(V té době nás bylo ve třídě 35 a zrovna nikdo nechyběl...)\newline
}
    \hlod{prof: "U tohoto příkladu si udělejte vykřičník. Tento příklad je strašně moc důležitý."\newline
třída: "To jste říkal i včera u toho minulého příkladu."\newline
prof: "Ale tenhle je opravdu moc důležitej. Tak si tam udělejte ten největší červený vykřičník!"\newline
(...)\newline
prof: "U tohoto příkladu si udělejte vykřičník."\newline
třída: "ZAS?!"\newline
(...)\newline
prof: "... tak, jednička, to bude příklad, který jste měli s vykřičníkem. Jaký to tedy bude?"\newline
třída: "Každý?!"\newline
}
    \hlod{(Středeční matematika, první hodina. Opět prořídlé řady ve třídě. Vchází Petr Zelina.)prof: "No, dobrý den. Proč nezavřeš?"Petr Zelina: "No (jako by to bylo úplně jasný), protože jdou další."\newline
}
    \hlod{prof: "... označíme tu rovnici kosočtvercem, nevím, proč je tam zrovna kosočtverec..."\newline
Tadeáš B. (po chvíli): "Co je tam ta piča?!"\newline
}
    \hlod{(Při písemce)\newline
prof: "Honzo (P.), kam se to díváš?!"\newline
Štěpán B.: "Tady toho fakt moc nenajdeš."\newline
}
    \hlod{prof: "Kdo to chápe? Nikdo? No, popravdě jsem doufal, že budu úspěšnější..."\newline
}
  \subsection{ Matematika - cvičení - prof. Aleš Kobza}
    \hlod{(Chybí Lucka B. a Matěj si sednul za Luckou H.)\newline
prof: "Dívám se, že Matěj provozuje zákon zachování Lucek!"\newline
}
    \hlod{(Zuzka maže tabuli.)\newline
Anička Nevídalová: "Máš pěknej zadek!"\newline
prof: "Děvčata, no ták. Já to slyším a pak se musím červenat."\newline
}
    \hlod{třída: "Pane profesore, proč nemůžeme mít třeba za deset plusek jeden bonusový bod?"\newline
prof: "No... Dobře, když to na mě neprásknete. Já bych to tak klidně rád udělal. Kolega Boucník ne. Takže já to taky nechci."\newline
}
    \hlod{prof: "... a teď to musíte ..."\newline
Zuzka Kuchařová: "... umocnit..."\newline
prof: "...na třetí! Enjoy!"\newline
}
    \hlod{(Prof. vejde do třídy, neřekne ani slovo a místo pozdravu pronese:)\newline
prof: "Jáááj!"\newline
}
    \hlod{(Zuzka sedí se Štěpánem)\newline
Zuzka K: "Já chci, abyste ho ode mě odsadil!"\newline
prof: "Zuzko, vy nepřestanete žvanit?"\newline
Zuzka K: "Dokud ho ode mě neodsadíte, budu žvanit."\newline
prof: "Beru to jako výzvu."\newline
}
    \hlod{prof: "Tohle si někam zapište. Zuzce došla slova."\newline
}
    \hlod{(Ondra B. měl spočítat 125-94 bez kalkulačky.)\newline
prof: "Kupecké počty! Vzdycháte jako v neslušném snímku!"\newline
}
    \hlod{prof: "Zuzko, pokračujte!"\newline
Zuzka K.: "Když já nevím, co s tím!"\newline
prof: "Ona NEVÍ co s tím! (radostný smích, úsměv, jak kdyby začínaly Vánoce) Paní chytrá NEVÍ! To si někam zapište."\newline
}
    \hlod{(Lucka H. popíše velmi neefektivním způsobem půlku tabule)\newline
prof: "Ano, vidím, že jste, Lucko, úspěšným absolventem kurzu úsporného psaní Zuzany Kuchařové."\newline
}
    \hlod{Zuzka K.: "Proč tady značíme řešení rovnice jako K a v normální matice jako P?"\newline
Lucka H.: "Protože K jako Kobza a P jako Pavel." \newline
}
    \hlod{(Matěj si od Lucky B. přesednul k její sestře, Peťce.)\newline
prof: "No, v září byl ten zasedací pořádek jinak. Aspoň můžu říct, že Matěj zůstal věrný rodině."\newline
}
  \subsection{ Fyzika - prof. Pavel Řehák }
    \hlod{Štěpán (u tabule): "Já si asi nepa... vlastně určitě si nepamatuju ten vzorec."\newline
}
    \hlod{(Prof. Kobza je nemocný)Holky: "Pane profesore, nevíte jak je panu profesorovi Kobzovi?"prof: "Prý je na antibiotikách a dívá se na Macha a Šebestovou."\newline
}
    \hlod{prof: "Ty seš..."\newline
Krab: "Úplnej retard."\newline
prof: "Děkuji."\newline
}
    \hlod{prof: "Vezmu si něco kulatého (rozhlíží se a nic nenajde)... Takže, třídní kniha se kutálí..."\newline
}
  \subsection{Anglický jazyk 2 - prof. Pavel Novák}
    \hlod{prof: "How was your weekend?"Anežka Janebová: "Horrible!"prof: "C'mon, I love horrible stories!"\newline
}
    \hlod{prof: "V chemii jsou důležité dvě věci: rovnice tvrdnutí malty a výroba alkoholu."\newline
}
    \hlod{prof: "Say hello to Mrs. Whateverhernameis."\newline
}
    \hlod{prof: "Lukáši! Úkol nemáš, slovíčka neumíš! Že já tě dám do lepší skupiny! (k profesorce Ševčíkové)!"\newline
}
  \subsection{Německý jazyk 1 - prof. Pavel Novák}
    \hlod{(Štěpán byl poslán s třídnicí do další skupiny)\newline
Nějaký student: "Dobrý den, nesu vám třídnici 1.A."\newline
prof: "Děkuji... Počkat, jak je možné, že třídnice je tu znova a ještě ke všemu dřív než Štěpán?!"\newline
}
  \subsection{Informatika - prof. Viera Hájková}
    \hlod{Lucka H.: "Paní profesorko, proč tam máte vy, vedle toho data, vhajkova, Anežka (Janebová) tam má xjaneb00 a já tam mám Meloun?!" \newline
}
  \subsection{Informatika - prof. Pavel Krejčí}
    \hlod{prof: "Blanka už to má, takže byste to měli mít všichni...!"\newline
}
    \hlod{(Někteří studenti vypomáhají na krajském kole SOČ tím, že hlídají školní elektroniku v učebnách. Blanka šla vyfasovat svůj notebook)\newline
prof: "Blanko?! Ty seš (smích) technická podpora?!"\newline
}
  \subsection{Chemie - prof. Veronika Kyasová}
    \hlod{prof: (cosi vykládá...) "... Štěpáne! Nespěte tam! Já tady neopakuju kvůli tomu, abyste spal na lavici! \newline
(za cca 5 min)  Štěpáne! To, že jsem vám řekla, že nemáte spát na lavici, neznamená, že smíte spát vsedě!"\newline
}
  \subsection{Zeměpis - prof. Eduard Pataki}
    \hlod{prof: "Štěpáne, jak je tam?"\newline
Štěpán Balážik: "Eh, vlhko a sucho."\newline
}
    \hlod{prof: "Teda, vy jste tak dobří! A víte proč? Máte výborného učitele!"\newline
}
    \hlod{(Prof. zkouší Michala a je to dost zoufalé...)\newline
prof: "No, tak nějaká další náboženství! Třeba v Izraeli!"\newline
Michal Š: "Jéžišmárjá."\newline
prof: "To tam právě není."\newline
}
  \subsection{Dějepis - prof. Alena Hanáková}
    \hlod{prof (postupně zvyšuje hlas) "Čísla panovníků se zásadně píší ŘÍMSKÝMI ČÍSLICEMI! \newline
BO - RÝS - KU!" \newline
}
    \hlod{Anežka J.: "Paní profesorko, nemohla bych se prosím nechat vyvolat zítra...?"\newline
prof: "No to ne! Vypadám snad jako Armáda spásy?"\newline
}
    \hlod{prof: "Dějepis je exaktní věda."\newline
}
    \hlod{Jirka P. : "... a ti se ubránili díky tomu, že je vzbudily husy."\newline
prof : "Výborně. Víte někdo, proč tam ty husy byly?"\newline
Tomáš K.: "Oni je používali jako alarm."\newline
}
  \subsection{Hudební výchova - prof. Irena Ambrozová}
    \hlod{prof: (opravuje písemku O. Svobody) "... Jan z Polžic, Bezdružic a Domažlic?! Ale jo, snažil se, beru to."\newline
}
    \hlod{(Hledá se třídní kniha, dokonce se pro ni bylo i ve výtvarce. Služba se vrátila...)\newline
Lucka H.: "Třídní knihu má prý Tomáš Kalvoda v aktovce."\newline
Tomáš K.: "Ajo."\newline
}
\section{Sexta}
  \subsection{Český jazyk - prof. Martina Literáková}
    \hlod{prof: "Možná to nevíte, ale chtěla bych vám oznámit, že Honza Priessnitz vyhrál školní kolo chemické olympiády kategorie A, to znamená, že byl lepší než většina maturantů!"\newline
Honza P.: (důrazně) "Všichni."\newline
}
    \hlod{(Omluvenky)\newline
prof: "Dandi?! Ranní nevolnost? To mívají těhotné ženy!"\newline
}
    \hlod{prof: "Blani, tak nám přečti svoji slohovku."\newline
Blanka M.: "Já jsem si vybrala jako téma trapasy. Kdybych ale věděla, že se to bude číst veřejně, tak bych si to nevybrala."\newline
}
  \subsection{Matematika - prof. Pavel Boucník}
    \hlod{Petr Zelina: "Já se omlouvám, že jdu pozdě. Nemohl jsem najít klíče a doma mě zamkli. Musel jsem čekat, než mě někdo vysvobodí..."\newline
prof: "A v kolikátým bydlíš patře?"\newline
}
    \hlod{prof: "Honzo!? Co to máš pod lavicí?! Okamžitě to vyhoď!"\newline
Štěpán B.:" Né, to je moje Bravíčko!"\newline
}
    \hlod{(Jako příklad na zobrazení máme náš bodový systém + a -)\newline
Honza P.:"Já myslím, že to je surjektivní, protože nikdo není takový tragéd, aby dostal nulu!"\newline
}
    \hlod{prof: "Tento příklad je za tři body. Pokud ho ale budete mít špatně, strhnu vám čtyři."\newline
}
    \hlod{prof: "Řešil jsem volitelné předměty. Všichni jste uspokojeni. Teda až na Peťku Bulantovou."\newline
}
    \hlod{prof: "Vy se učíte tak do dějepisu!"\newline
třída: "My musíme, že jo."\newline
prof: "To všechno najdete na internetu, prakticky okamžitě!"\newline
třída: "Tak to řekněte prof. Hanákové!"\newline
prof (zničeně): "Já jí to říkám neustále."\newline
}
    \hlod{prof: "Eulerovo číslo je dvě celé sedm jedna osm dva osm jedna ... Jo, umím to na dvacet desetinných míst. Když mě učil pan ředitel, tak říkal, že on sám to umí na dvacet desetinných míst. Tak jsem se to naučil taky. Ale je to úplně k ničemu."\newline
}
    \hlod{prof: "Stačí mi, když bude mít zkoušení hlavu a patu."\newline
ze třídy: "Nebo prsa a zadek."\newline
}
    \hlod{(O vodáckém kurzu)\newline
prof: "Když si na sebe dáte neopren a pláštěnku, vytvoříte mikroklíma. Sice smrádeček, ale teplíčko."\newline
}
  \subsection{Matematika - cvičení - prof. Aleš Kobza}
    \hlod{prof: "Máte nějaké otázky?"\newline
Zuzka K.: "Kdo byla ta slečna a proč jí tykáte?!"\newline
}
    \hlod{prof: "Já jsem náhodou během svých studentských let chodil na brigády kopat hezky krumpáčem. A spoustu jsem se toho naučil. Mimo jiné ukrajinsky."\newline
}
  \subsection{Fyzika - prof. Pavel Řehák }
    \hlod{prof: "Na referáty tady jsou... zážehové motory, spalovací motory, motory u raketoplánů..."\newline
Zuzka K.: "Nemáte tam něco víc holčičího?"\newline
prof: "No, je tu chladnička."\newline
Zuzka K.: "Tak tu beru. To je jediný slovo, kterýmu tam rozumím."\newline
}
  \subsection{Informatika - prof. Viera Hájková}
    \hlod{prof: "... opravdu, jedna z nejdůležitějších věcí je umět pochopit, co se po vás chce. Nebudu jmenovat pana zástupce, ale já často opravdu nevím, co po mně chce."\newline
Lucka H.:"To byste měla s námi jít do matiky někdy, abyste pochopila myšlenkové pochody profesora Boucníka!"\newline
prof: "Ne. Pokud něco nechci v životě pochopit, pak jsou to právě myšlenkové pochody profesora Boucníka."\newline
}
  \subsection{Chemie - prof. Veronika Kyasová}
    \hlod{prof: "Štěpáne, já vás asi přijdu něčím bacit."\newline
Štěpán: "A proč?"\newline
prof: "Jen tak, abych si udělala radost a měla hezký den."\newline
Štěpán: "Tak fajn. Rád rozdávám radost."\newline
}
  \subsection{Dějepis - prof. Alena Hanáková}
    \hlod{prof.: "... tvrdí, že člověk už za svého života směřuje k zatracení. Když člověk krade, vraždí nebo nedává pozor v dějepisu, bude zatracen."\newline
}
    \hlod{(Během výkladu se objevilo nějak moc Otakarů...)\newline
Lucka H.: "Aha! Takže ten král akorát pojmenoval svoje dítě po svém strýci...?"\newline
prof: "Ne! Lucie, udělejte si jasno v příbuzenských vztazích! Teda máte pravdu!"\newline
}
    \hlod{(Adam je vyvolán k tabuli a popotahuje rýmu).\newline
prof: "Adame, máte kapesník?"\newline
Adam Č.: "Bohužel, právěže nemám."\newline
prof: "Já vám jeden tady půjčím." (Adam smrká) "Doufám, že to umíte a nebudete pak plakat a já vám nebudu muset dát další."\newline
}
    \hlod{prof: "Pan zástupce, kterého máte tuším z matematiky, na mě ušil takovou kulišárnu, které jste si určitě všimli na suplování. On po mně chce, abych šla suplovat k těm prckům na Příční!"\newline
.......\newline
prof: "... vidíte, toto je taková krásná látka, o té bychom si mohli povídat dlouho..., pásy cudnosti... jenomže to nám bohužel vzal profesor Boucník."\newline
}
    \hlod{Štěpán B.: "Paní profesorko, už jsem si vybral adekvátní termín zkoušení z dějepisu."\newline
prof: "Ano?"\newline
Štěpán B.: "Prosím, pátek třináctého."\newline
}
  \subsection{Společenské vědy - prof. Markéta Chalupníková}
    \hlod{prof: "Zuzko, uměla jste to, učila jste se. Jedna."\newline
Zuzka K (cestou k lavici, mrmlá): "Neučila, ale jsou to společenský vědy, no."\newline
}
    \hlod{prof: "V adolescenci se rozvíjí abstraktní myšlení, takže ty blbosti, co se učí v matematice..."\newline
}
    \hlod{prof: "Pyknici. To jsou takoví ti typičtí matematici nebo informatici, hubený, pokřivený, velká hlava, ohnutý ručky... Jé, já tu teď možná někoho urazila..."\newline
}
  \subsection{Výtvarná výchova - prof. Michal Hubáček}
    \hlod{prof: "Kdo by měl zájem o exkurzi do Maunthausenu?"\newline
Peťka B.: "Já."\newline
prof: "A co sestra?"\newline
Peťka B.: "Pro tu jen jednosměrnou jízdenku."\newline
}
\section{Septima}
  \subsection{Český jazyk - prof. Martina Literáková}
    \hlod{prof: "Pošlete třídnici do matematiky, protože náš kolega iluminátor nezapsal."\newline
}
    \hlod{prof: "Tadeáši, já ti dám zápisky! Co je tam to prasátko s vrtícím ocáskem?"\newline
}
    \hlod{(Krab obhajuje ZMP)\newline
Krab: "... on je opravdu velice starý, je mu třeba 45 let..."\newline
prof: "Přemýšlím, jestli si už mám jít teda kopat hrob...?"\newline
}
    \hlod{(Dandi o surrealistickém textu)\newline
Dandi: "Abych byl upřímný, mně to přijde jako výklad profesora Boucníka: pátý přes devátý, nechápu to..."\newline
}
  \subsection{Matematika - prof. Pavel Boucník}
    \hlod{(Na začátku června)\newline
prof: "Tak a zítra budu zkoušet ty, co nebyli. Musíme dát nějaká znamínka!"\newline
Lucka H.:"Pane profesore, tak mi tam rovnou napište dvě mínuska. Mně to stejně body nezmění, furt budu mít jedničku."\newline
prof: "Počkej, to teda ne! Co kdybych tě vyzkoušel znova?!"\newline
Lucka H.:"Tak byste mi dal další dvě mínuska. A opět by mi to nevadilo. Takhle byste mohl klidně zkoušet 35 krát a nic by to neměnilo. Tak mě aspoň nemusíte zkoušet."\newline
prof: "Počkej, počkej, počkej. Tak to ne! To by znamenalo, že celý ten systém je špatný!"\newline
Ohlušující potlesk třídy.\newline
}
  \subsection{Matematika - prof. Aleš Kobza}
    \hlod{(Čtvrtletní písemka)\newline
prof: "Prosím vás, aspoň tu spolu přede mnou neflirtujte..."\newline
Tomáš K.:"To máme jako flirtovat s váma?"\newline
...\newline
prof: "Anežko! Nekoukejte tam. Pojďte do první lavice. Šup!"\newline
Anežka J.:"Proč já, a ne Matěj?"\newline
šeptem zepředu: "Aby se mohl dívat na tebe."\newline
... \newline
Petr S.:(mrmlá si): "... určete pravděpodobnost... CO?!"\newline
...\newline
prof: "Za tohle bych měl dostat nějaký wellness..."\newline
Štěpán B.:"Na Kanáry, třeba, co?"\newline
}
    \hlod{(Lucka B. počítá týdeňák u tabule)\newline
prof: "No tak, Lucko, dáme to spolu dohromady."\newline
Lucka B.(vyzývavě): "Jóóó?"\newline
}
    \hlod{prof: "... to je jak Nelinka. Tudle se u večeře zeptala "Maminko, a můžu říkat doprdele?""\newline
}
    \hlod{prof: "Anežko, vy tam zase datlíte do mobilu, co?"\newline
Anežka:"Pane profesore, proč se mě na to ptáte, když je to evidentní?"\newline
Petr Sevelda: "Tý jo, perfektní ženskej protiútok."\newline
}
  \subsection{Fyzika - prof. Pavel Řehák}
    \hlod{prof: "No jó, zmizla televize."\newline
Tomáš K.: "Kdy byla naposledy použita?"\newline
prof: "To bylo minulou... válku."\newline
}
    \hlod{(Štěpán přišel v půlce hodiny a po deseti minutách odchází).\newline
prof: "To byla teda krátká návštěva."\newline
Štěpán B.:"Já jsem sem šel více méně dobrovolně. Já sem vůbec nemusel chodit."\newline
prof: "A celý těch deset minut jste litoval, že jste přišel, co?"\newline
Štěpán B.:"Jo."\newline
}
  \subsection{Biologie - prof. Iva Kubištová}
    \hlod{prof: "Další formy antikoncepce?"\newline
Lucka H: "Abstinence!"\newline
(smích třídy)\newline
Blanka M.: "Nám to, náhodou, s Luckou zatím vychází...!"\newline
}
    \hlod{(Slečna posluchačka, mimochodem dcera paní profesorky třídní, zapíná dataprojektor. Musela si vylézt na lavici)\newline
Krab: "No teda! V botech!!!"\newline
posluchačka: "Neříkejte to mamince..."\newline
}
  \subsection{Dějepis - prof. Alena Hanáková}
    \hlod{prof: "Dobrovolník k tabuli? Není? Tak, je 25. 9., čili 25. Sedlák. Ne, ten má individuál. Tak 25 - 9, to je 16, Krumlová, taky individuál! Tak 16 - 9, je 7, Buriánek."\newline
třída: "Ten je ve Valencii."\newline
prof : "Boček?"\newline
třída: "Valencie."\newline
prof: "Blahynka, ten chybí. Tak nic. (začne rozpočítávat). Ententýky, dva špalíky, čert vyletěl z elektriky, boule byla veliká, jako celá Afrika." (David Paleček)\newline
David Paleček: "Já už ale byl."\newline
prof: "Tak nic, Jirko, pojďte!"\newline
}
    \hlod{Zuzka K.:"Paní profesorko, to národní obrození budete zkoušet?"\newline
prof: "No..., budu, ale já ho od té doby, co jsem psala diplomovou práci na národní obrození na Moravě, tak nemám ráda!"\newline
}
  \subsection{Společenské vědy - prof. Chalupníková}
    \hlod{Honza P.:"No a v tom druhém kole se rozhoduje mezi dvěma kandidáty, to znamená, že aby byl jeden z nich zvolen, tak musí mít nadpoloviční většinu hlasů."\newline
prof: "Hm... Já jsem to vždycky chápala tak, že stačí, když jeden kandidát má víc hlasů než ten druhý, to jsem nevěděla, že musí získat dokonce nadpoloviční většinu..."\newline
}
    \hlod{Lucka H.:"Můžu mít dotaz? Paní profesorko, ne, že bych něco z toho plánovala, ale... V Česku je zneužití dětí trestný, že jo. Co kdybych, ale, vyrazila někam do ciziny, kde to je povolený? Porušila bych tím něco?“ \newline
Chalupníková: "Ježiš... no... Nemyslím si, že taková země ale existuje..."\newline
Tomáš K.:"Lucko, klid, kdyby taková byla, už byste tam dávno jeli na vodu."\newline
...\newline
Lucka H:"Nebo druhej příklad, že jo. Sexuální aktivity v rámci České republiky jsou povolené od 15 let. Ale na Maltě je to 18. Je tedy možné, aby si přijeli dva šestnáctiletí Malťani užít beztrestně do Česka?"\newline
Chalupníková: "Ale co když by pak otěhotněla?"\newline
Lucka H:"Jenže ono těhotenství zakázaný není, tam je jenom zakázaná ta činnost, která těhotenství předchází. A tu vykonali v Česku, takže by to neměl být problém, ne?"\newline
(Po chvíli zamyšlení): Chalupníková: "Lucko, tohle opravdu nevím. To budete muset vyzkoušet sama."\newline
}
  \subsection{Školní výlety, přestávky a jiné víceméně legální akce}
    \hlod{prof. Urban: "O facebooku, o tom mi ani nevykládejte. Nedávno jsem se na něj zaregistroval, abych zjistil, co to umí. Horší bylo, že jsem použil email mé ženy. Potom jí začaly chodit zprávy, že si mě chce přidat několik žen, které jsem v životě neviděl. Jenže vysvětlujte to doma!"\newline
}
    \hlod{prof. Urban: "Teď vám řeknu, proč byste měli chodit vepředu s učitelem (jako obvykle prof. šel o hodně před námi).  Když jsem tu asi před 5 lety byl, tak tady dole, tam je nuda pláž. Šel jsem a byly tam nějaké dvě nahé slečny. Říkám jim, že mně to teda nevadí, ale za mnou, že jde 60 studentů. Tak mi poděkovaly a spěšně odešly. Kdyby tehdy šli studenti se mnou, viděli by ty slečny, takhle je neviděli."\newline
}
  \subsection{Výměnné pobyty}
    \hlod{Zvolská: "Prosím vás, čeká vás víkend ve vašich hostitelských rodinách. Určitě si ho užijte, hlavně jim pak poděkujte. V případě jakéhokoliv sebemenšího problému mi ihned volejte. Moje telefonní číslo máte. Takže pamatujte, hlavně opatrně..."\newline
Řihánková: (přeruší prof. Zvolskou): "Prostě se neožerte!"\newline
}
  \subsection{Malino Brdo, Silvestr 2014}
    \hlod{Veselá: "To je taková španělská tradice, že si hrozny namočíme do... uhm ... nealkoholické vodky..."\newline
}
  \subsection{Vodácký kurz, 2015, Moravice}
    \hlod{Boucník: "Prosím vás, teď si rozdáme vesty bijektivně. Dvě pádla budou vždy přiřazena k jedné lodi. Ježiši, to je nádhera."\newline
}
\section{Oktáva}
  \subsection{Český jazyk - prof. Martina Literáková}
    \hlod{prof: „Neexistuje, že si dáte v Praze po obědě třeba pivo.“\newline
Honza P.:“Ááh!“\newline
prof: „Co se děje, Honzo?“\newline
Honza P.:“Já si jenom vzpomněl, jak je to pivo po obědě fakt dobrý.“\newline
}
  \subsection{Matematika - prof. Pavel Boucník }
    \hlod{Boucník: „Ale tak děcka, přece nematurujete poprvé.“\newline
}
    \hlod{Boucník: „Na to nemáme čas!“\newline
}
    \hlod{Boucník: „No, Blanko, ještě bys odmaturovala…“\newline
Blanka M.: „To nevadí, na podzim jdu stejně kvůli chemii.“\newline
}
  \subsection{Matematika - prof. Aleš Kobza }
    \hlod{(Tomáš Medek je drzý)\newline
prof: „Prosím vás, kdybyste ho někdo zabil, já to na vás neřeknu.“\newline
}
  \subsection{Fyzika - prof. Pavel Řehák}
    \hlod{(Ruch ve třídě)\newline
Řehák: „Co se děje?“\newline
Lucka H.: „Tam v tom okně je nahej chlap…“\newline
Řehák: (naznačí sundání trika) „Když se tady vysleču, budete mi věnovat stejnou pozornost?“\newline
}
  \subsection{Společenské vědy - prof. Chalupníková}
    \hlod{prof: „… a ten tvrdil, že populace roste geometrickou řadou, ale potrava pouze aritmetickou.“\newline
Dandi: „Paní profesorko, tomuto já nerozumím. Chci požádat o vysvětlení geometrické řady.“\newline
prof: „Ha! A to já tu náhodou mám. Takže geometrická řada je jedna, dva, čtyři, osm…“\newline
}
    \hlod{}
