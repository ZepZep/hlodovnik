\section*{Kvinta}
  \subsection*{Český jazyk -- prof. Martina Literáková}
    \hlod{\textit{prof}: \uv{Které z Desatera přikázání se už nyní hodně porušuje?}\newline
    \textit{Tomáš B}: \uv{Nesesmilníš!}\newline
    \ldots{}\newline
    (\textit{Rozbor básně. Došli jsme k jistému tématu})\newline
    \textit{prof}: \uv{No a co se stalo pak\ldots{}?}\newline
    \textit{Tomáš B}: \uv{Já myslím, že sesmilnili.}}
    \hlod{(\textit{Prof. shrnuje první pololetí})\newline
    \textit{prof}: \uv{\ldots{} Štěpáne, myslím si, že ses z té důtky ponaučil, jsem za to velice ráda\ldots{}}\newline
    \textit{Štěpán B.}: \uv{No, mě jenom překvapuje, že ačkoli mě pan ředitel označil za zloděje, tak jsou mi stále svěřovány až desetitisícové částky.} (\textit{Štěpán vybíral peníze na lyžák}).}
    \hlod{\textit{prof}: \uv{Kdy můžeme někomu skočit do řeči a není to neslušné?}\newline
    \textit{Dandi}: \uv{No, třeba když hoří.}}
  \subsection*{Matematika -- prof. Pavel Boucník}
    \hlod{\textit{prof (\textit{naštvaně})}: \uv{Jestli je vám to jasný\ldots{} (\textit{uvědomí si, že nemá co dodat})\ldots{} tak stejně mlčte!}}
    \hlod{\textit{prof}: (\textit{mluví o všeobecných třídách}) \uv{No, pro ty normální je to \ldots{}tím  teda nechci říct, že vy nejste normální\ldots{} No, vlastně (\textit{podívá se nás}) \ldots{} Prostě, pro ty normální by to bylo takhle!}}
    \hlod{\textit{prof}: \uv{Dělit nulou nelze, takže my si to samozřejmě vyzkoušíme.}}
    \hlod{\textit{prof}: \uv{Kdo to chápe?} \newline
    \textit{třída}: (\textit{přihlásí se asi 5 lidí})\newline
    \textit{prof}: \uv{Výborně, skoro všichni!}}
    \hlod{\textit{prof}: \uv{Tak, kdo to má správně?}\newline
    \textit{třída}: (\textit{přihlásí se 7 lidí})\newline
    \textit{prof}: \uv{Výborně, to jsou dvě třetiny!}\newline
    (\textit{V té době nás bylo ve třídě 35 a zrovna nikdo nechyběl\ldots{}})}
    \hlod{\textit{prof}: \uv{U tohoto příkladu si udělejte vykřičník. Tento příklad je strašně moc důležitý.}\newline
    \textit{třída}: \uv{To jste říkal i včera u toho minulého příkladu.}\newline
    \textit{prof}: \uv{Ale tenhle je opravdu moc důležitej. Tak si tam udělejte ten největší červený vykřičník!}\newline
    (\textit{\ldots{}})\newline
    \textit{prof}: \uv{U tohoto příkladu si udělejte vykřičník.}\newline
    \textit{třída}: \uv{ZAS?!}\newline
    (\textit{\ldots{}})\newline
    \textit{prof}: \uv{\ldots{} tak, jednička, to bude příklad, který jste měli s vykřičníkem. Jaký to tedy bude?}\newline
    \textit{třída}: \uv{Každý?!}}
    \hlod{(\textit{Středeční matematika, první hodina. Opět prořídlé řady ve třídě. Vchází Petr Zelina.})prof: \uv{No, dobrý den. Proč nezavřeš?}Petr Zelina: \uv{No (\textit{jako by to bylo úplně jasný}), protože jdou další.}}
    \hlod{\textit{prof}: \uv{\ldots{} označíme tu rovnici kosočtvercem, nevím, proč je tam zrovna kosočtverec\ldots{}}\newline
    Tadeáš B. (\textit{po chvíli}): \uv{Co je tam ta piča?!}}
    \hlod{\textit{Petr S}: \uv{Ale pane profesore, když za to x dosadíme 1, tak ta věta nemusí platit.}\newline
    \textit{prof}: \uv{Eh (\textit{začne o tom přemýšlet, po asi 5 minutách}). Né, moment. Tohle určitě platí. Já jenom určitě blbě přemýšlím!}}
    \hlod{(\textit{Hádáme se o tom, jestli daná věta platí. První část důkazu ekvivalence je označena pouze jako \uv{zřejmé}})\newline
    \textit{prof}: \uv{No, tak to musí být chyba v tom důkazu. (\textit{po zkontrolování}): Ne, je dobře.}\newline
    \textit{Bára V.}: \uv{Třeba je chyba v tom }zřejmé\uv{ .}}
    \hlod{(\textit{Při písemce})\newline
    \textit{prof}: \uv{Honzo (\textit{P.}), kam se to díváš?!}\newline
    \textit{Štěpán B.}: \uv{Tady toho fakt moc nenajdeš.}}
    \hlod{\textit{prof}: \uv{Kdo to chápe? Nikdo? No, popravdě jsem doufal, že budu úspěšnější\ldots{}}}
    \hlod{(\textit{Píšeme písemku. Tadeáš se téměř ihned po dodiktování zvedne k odevzdání písemky})\newline
    \textit{prof}: (\textit{nadšeně z geniality svého studenta}) \uv{Tadeáši! Ty už to máš?!}\newline
    Tadeáš B. (\textit{naprosto klidně}). \uv{Ne, já to prostě nevím.}}
  \subsection*{ Matematika -- cvičení -- prof. Aleš Kobza}
    \hlod{\textit{prof}: \uv{Spěchejme pomalu!}}
    \hlod{\textit{prof}: \uv{Vy počítáte jak exministr Kalousek!}}
    \hlod{(\textit{Chybí Lucka B. a Matěj si sednul za Luckou H.})\newline
    \textit{prof}: \uv{Dívám se, že Matěj provozuje zákon zachování Lucek!}}
    \hlod{(\textit{Zuzka maže tabuli.})\newline
    \textit{Anička Nevídalová}: \uv{Máš pěknej zadek!}\newline
    \textit{prof}: \uv{Děvčata, no ták. Já to slyším a pak se musím červenat.}}
    \hlod{\textit{třída}: \uv{Pane profesore, proč nemůžeme mít třeba za deset plusek jeden bonusový bod?}\newline
    \textit{prof}: \uv{No\ldots{} Dobře, když to na mě neprásknete. Já bych to tak klidně rád udělal. Kolega Boucník ne. Takže já to taky nechci.}}
    \hlod{\textit{prof}: \uv{\ldots{} a teď to musíte \ldots{}}\newline
    \textit{Zuzka Kuchařová}: \uv{\ldots{} umocnit\ldots{}}\newline
    \textit{prof}: \uv{\ldots{}na třetí! Enjoy!}}
    \hlod{\textit{prof}: \uv{4. Newtonův pohybový zákon. Když to nejde silou F, jde to silou 2F.}}
    \hlod{(\textit{Prof. vejde do třídy, neřekne ani slovo a místo pozdravu pronese:})\newline
    \textit{prof}: \uv{Jáááj!}}
    \hlod{(\textit{Zuzka sedí se Štěpánem})\newline
    \textit{Zuzka K}: \uv{Já chci, abyste ho ode mě odsadil!}\newline
    \textit{prof}: \uv{Zuzko, vy nepřestanete žvanit?}\newline
    \textit{Zuzka K}: \uv{Dokud ho ode mě neodsadíte, budu žvanit.}\newline
    \textit{prof}: \uv{Beru to jako výzvu.}}
    \hlod{\textit{prof}: \uv{Tohle si někam zapište. Zuzce došla slova.}}
    \hlod{prof (\textit{chce rozdávat pluska}) \uv{Kdo to měl správně? VŠICHNI! To mě těší! To budou určitě z 25 bodovky samé\ldots{}}\newline
    \textit{kdosi}: \uv{\ldots{}jedničky.} }
    \hlod{(\textit{Ondra B. měl spočítat 125-94 bez kalkulačky.})\newline
    \textit{prof}: \uv{Kupecké počty! Vzdycháte jako v neslušném snímku!}}
    \hlod{\textit{prof}: \uv{Zuzko, pokračujte!}\newline
    \textit{Zuzka K.}: \uv{Když já nevím, co s tím!}\newline
    \textit{prof}: \uv{Ona NEVÍ co s tím! (\textit{radostný smích, úsměv, jak kdyby začínaly Vánoce}) Paní chytrá NEVÍ! To si někam zapište.}}
    \hlod{(\textit{Lucka H. popíše velmi neefektivním způsobem půlku tabule})\newline
    \textit{prof}: \uv{Ano, vidím, že jste, Lucko, úspěšným absolventem kurzu úsporného psaní Zuzany Kuchařové.}}
    \hlod{\textit{prof}: \uv{Přemýšlím, jestli mám černé kalho\ldots{}}\newline
    \textit{Zuzka K.}: \uv{JO! Máte!}}
    \hlod{\textit{prof}: \uv{Zuzko, jaké by bylo vaše vysněné povolání?}\newline
    \textit{Zuzka K.}: \uv{Spalovač mrtvol.}}
    \hlod{\textit{Zuzka K.}: \uv{Proč tady značíme řešení rovnice jako K a v normální matice jako P?}\newline
    \textit{Lucka H.}: \uv{Protože K jako Kobza a P jako Pavel.} }
    \hlod{(\textit{Matěj si od Lucky B. přesednul k její sestře, Peťce.})\newline
    \textit{prof}: \uv{No, v září byl ten zasedací pořádek jinak. Aspoň můžu říct, že Matěj zůstal věrný rodině.}}
    \hlod{(\textit{Prof. čte zadání příkladu})\newline
    \textit{prof}: \uv{Poměr mezi muži byl\ldots{} CO?!} }
  \subsection*{ Fyzika -- prof. Pavel Řehák }
    \hlod{Štěpán (\textit{u tabule}): \uv{Já si asi nepa\ldots{} vlastně určitě si nepamatuju ten vzorec.}}
    \hlod{(\textit{Prof. Kobza je nemocný})Holky: \uv{Pane profesore, nevíte jak je panu profesorovi Kobzovi?}prof: \uv{Prý je na antibiotikách a dívá se na Macha a Šebestovou.}}
    \hlod{\textit{prof}: \uv{Ty seš\ldots{}}\newline
    \textit{Krab}: \uv{Úplnej retard.}\newline
    \textit{prof}: \uv{Děkuji.}}
    \hlod{\textit{prof}: \uv{Pánové, Tadeáši\ldots{}}}
    \hlod{\textit{prof}: \uv{Vezmu si něco kulatého (\textit{rozhlíží se a nic nenajde})\ldots{} Takže, třídní kniha se kutálí\ldots{}}}
  \subsection*{Anglický jazyk 2 -- prof. Pavel Novák}
    \hlod{\textit{prof}: \uv{How was your weekend?}Anežka Janebová: \uv{Horrible!}prof: \uv{C'mon, I love horrible stories!}}
    \hlod{\textit{prof}: \uv{V chemii jsou důležité dvě věci: rovnice tvrdnutí malty a výroba alkoholu.}}
    \hlod{\textit{prof}: \uv{Say hello to Mrs. Whateverhernameis.}}
    \hlod{\textit{prof}: \uv{Lukáši! Úkol nemáš, slovíčka neumíš! Že já tě dám do lepší skupiny! (\textit{k profesorce Ševčíkové})!}}
  \subsection*{Německý jazyk 1 -- prof. Pavel Novák}
    \hlod{(\textit{Štěpán byl poslán s třídnicí do další skupiny})\newline
    \textit{Nějaký student}: \uv{Dobrý den, nesu vám třídnici 1.A.}\newline
    \textit{prof}: \uv{Děkuji\ldots{} Počkat, jak je možné, že třídnice je tu znova a ještě ke všemu dřív než Štěpán?!}}
    \hlod{(\textit{řešíme, co budeme dělat o podzimních prázdninách})\newline
    \textit{Zuzka Kuchařová}: \uv{Ich gehe zum konzentrazion kamp.}\newline
    \textit{prof}: \uv{Óh, wohin?}\newline
    \textit{Zuzka Kuchařová}: \uv{Zur Jeseníky\ldots{} Jakože na soustředění, ne do koncentráku!}\newline
    \textit{prof}: \uv{\ldots{} schválně, jak poznáte vy, co je zelenina a co ovoce?}\newline
    \textit{Lucka H.}: \uv{Ovoce je to, co mi chutná víc.}}
  \subsection*{Informatika -- prof. Viera Hájková}
    \hlod{\textit{Lucka H.}: \uv{Paní profesorko, proč tam máte vy, vedle toho data, vhajkova, Anežka (\textit{Janebová}) tam má xjaneb00 a já tam mám Meloun?!} }
    \hlod{\textit{prof}: \uv{Ne každý muž snese, když je jeho žena chytřejší než on sám. Já měla ale štěstí, mému muži to nevadí.}}
  \subsection*{Informatika -- prof. Pavel Krejčí}
    \hlod{\textit{prof}: \uv{Ticho, ticho! Blanka přemýšlí!}}
    \hlod{\textit{prof}: \uv{Blanka už to má, takže byste to měli mít všichni\ldots{}!}}
    \hlod{(\textit{Někteří studenti vypomáhají na krajském kole SOČ tím, že hlídají školní elektroniku v učebnách. Blanka šla vyfasovat svůj notebook})\newline
    \textit{prof}: \uv{Blanko?! Ty seš (\textit{smích}) technická podpora?!}}
  \subsection*{Chemie -- prof. Veronika Kyasová}
    \hlod{\textit{prof}: \uv{Já se vám do písemky vždycky snažím dát lehčí příklady, než jsme dělali na hodině.}\newline
    \textit{třída}: \uv{To jste hodná. Profesor Boucník to dělá přesně naopak.}}
    \hlod{\textit{prof}: (\textit{cosi vykládá\ldots{}}) \uv{\ldots{} Štěpáne! Nespěte tam! Já tady neopakuju kvůli tomu, abyste spal na lavici! \newline
    (\textit{za cca 5 min})  Štěpáne! To, že jsem vám řekla, že nemáte spát na lavici, neznamená, že smíte spát vsedě!}}
  \subsection*{Zeměpis -- prof. Eduard Pataki}
    \hlod{(\textit{P. prof. pošle Štěpána k němu do kabinetu, aby mu přinesl: notebook, napájecí kabel a tyč na zapínání dataprojektoru})\newline
    prof. (\textit{na Štěpána}) "\ldots{} a projev při plnění tvého úkolu dostatečnou inteligenci!"\newline
    (\textit{Štěpán je už skoro ve dveřích, najednou se otočí})\newline
    Štěpán Balážik : "Nenarazím tam na nic, na co bych narazit neměl, že?"\newline
    prof. (\textit{v té době s rýmou}) "Maximálně na posmrkané kapesníky\ldots{}, na hodně posmrkaných kapesníků\ldots{}!"\newline
    (\textit{Štěpán se vrací a nese: notebook, napájecí kabel, tyč na zapínání dataprojektoru a ještě nese nějaké papíry\ldots{}})\newline
    \textit{Štěpán Balážik}: "Při plnění svého úkolu jsem projevil zřejmě více inteligence, než bylo potřeba\ldots{} Donesl jsem ještě tyto papíry, na kterých bylo "1.A rozdat". \newline
    \textit{prof. }: "To jsem po tobě ale nechtěl, takže je dones zpátky!"}
    \hlod{ (\textit{Den voleb do Poslanecké sněmovny 2013})\newline
    \textit{prof}: \uv{Dnes máme poslední demokratický pátek.}}
    \hlod{\textit{prof}: \uv{Štěpáne, jak je tam?}\newline
    \textit{Štěpán Balážik}: \uv{Eh, vlhko a sucho.}}
    \hlod{\textit{prof}: \uv{Teda, vy jste tak dobří! A víte proč? Máte výborného učitele!}}
    \hlod{\textit{prof}: \uv{Řekněme, že Pataki je geologický činitel. Takže jestli si Pataki vezme kladivo a bude bušit do skály, tak se ze skály stane zvětralina.}}
    \hlod{\textit{prof}: \uv{Máte tuto knihu?}\newline
    \textit{třída}: \uv{Ano.}\newline
    \textit{prof}: \uv{Učí se vám z ní dobře?}\newline
    \textit{třída}: \uv{Ano.}\newline
    \textit{prof}: \uv{Líbí se vám ty obrázky?}\newline
    \textit{třída}: \uv{Ano.}\newline
    \textit{prof}: \uv{Jste s ní spokojení?}\newline
    \textit{třída}: \uv{Ano.}\newline
    \textit{prof}: \uv{Rozpadá se vám?}\newline
    \textit{třída}: \uv{Ne.}\newline
    \textit{prof}: \uv{To přijde.}}
    \hlod{\textit{prof}: \uv{\ldots{} svět se dělí na lidi a ženy.}}
    \hlod{(\textit{Prof. zkouší Michala a je to dost zoufalé\ldots{}})\newline
    \textit{prof}: \uv{No, tak nějaká další náboženství! Třeba v Izraeli!}\newline
    \textit{Michal Š}: \uv{Jéžišmárjá.}\newline
    \textit{prof}: \uv{To tam právě není.}}
    \hlod{(\textit{Petr Z. má v prezentaci \uv{vzdušný výr}})\newline
    \textit{prof}: \uv{Á, pán je ornitolog!}}
  \subsection*{Dějepis -- prof. Alena Hanáková}
    \hlod{prof (\textit{postupně zvyšuje hlas}) \uv{Čísla panovníků se zásadně píší ŘÍMSKÝMI ČÍSLICEMI! \newline
    BO - RÝS - KU!} }
    \hlod{\textit{prof}: \uv{Zelina je kde? Sbor? Na toho se těším. Poslední, kdo nebyl zkoušený. Toho si vychutnám! Klidně mu to vyřiďte.}}
    \hlod{\textit{Anežka J.}: \uv{Paní profesorko, nemohla bych se prosím nechat vyvolat zítra\ldots{}?}\newline
    \textit{prof}: \uv{No to ne! Vypadám snad jako Armáda spásy?}}
    \hlod{\textit{prof}: \uv{Dějepis je exaktní věda.}}
    \hlod{\textit{prof}: \uv{Jak byste se ohodnotil?}\newline
    \textit{Matěj Ž.}: \uv{Za pět?}\newline
    \textit{prof}: \uv{Alespoň na něčem jsme se shodli.}}
    \hlod{\textit{prof}: \uv{Mám pro vás dobrou zprávu. Jeden student z 1.C ukončil studium na této škole\ldots{} (\textit{třída vybuchla smíchy}), takže tu mám jednu učebnici a jeden atlas. Kdo je chce?}}
    \hlod{\textit{prof}: \uv{Tak mi to přečtěte z té mapy na stěně! Dělejte! Je to tam velkým písmem!}\newline
    David P. (\textit{sedí nejblíž}) \uv{\ldots{} uhm. Eh. Co?!}\newline
    \textit{prof}: \uv{Matematická třída neumí číst, to je výborné. (\textit{jde k mapě}). Jé, ono je to tam řecky, tak se trochu omlouvám.}}
    \hlod{Jirka P. : \uv{\ldots{} a ti se ubránili díky tomu, že je vzbudily husy.}\newline
    \textit{prof }: \uv{Výborně. Víte někdo, proč tam ty husy byly?}\newline
    \textit{Tomáš K.}: \uv{Oni je používali jako alarm.}}
    \hlod{\textit{prof}: \uv{A kde tuto taktiku použili?}\newline
    \textit{Dandi}: \uv{V bitvě.}}
  \subsection*{Tělesná výchova dívky -- prof. Hana Vladíková}
    \hlod{(\textit{S holkama vcházíme do velké tělocvičny. Skrz sklo na dveřích na nás ťuká student z 1.C, protože chce projít a přímo přede dveřmi stojí prof. Vladíková.})\newline
    \textit{student}: \uv{Oh, děkuju. Nechtěl jsem vás povalit.}\newline
    \textit{prof}: (\textit{posměšně}) \uv{Zkus to!}}
    \hlod{(\textit{Jdeme do parku běhat 100 m})\newline
    \textit{prof}: \uv{Tak. Toto je start. Cíl bude tam, u branky.}\newline
    \textit{všechny holky}: \uv{To není sto metrů\ldots{}}\newline
    \textit{prof}: \uv{Prosím vás, je! Už to tady běhám dost dlouho, tak to vím,ne?}\newline
    (\textit{První dvě holky se po dlouhé debatě nachystaly na start})\newline
    \textit{Peťka B.}:\uv{Připravit. Pozor. Start!}\newline
    \textit{sprinterky}: \uv{Paní profesorko, tak jaký máme čas?}\newline
    \textit{prof}: \uv{Hm. Doběhly jste ve světovém rekordu\ldots{}}}
    \hlod{}
  \subsection*{Hudební výchova -- prof. Irena Ambrozová}
    \hlod{\textit{prof}: (\textit{opravuje písemku O. Svobody}) \uv{\ldots{} Jan z Polžic, Bezdružic a Domažlic?! Ale jo, snažil se, beru to.}}
    \hlod{(\textit{Hledá se třídní kniha, dokonce se pro ni bylo i ve výtvarce. Služba se vrátila\ldots{}})\newline
    \textit{Lucka H.}: \uv{Třídní knihu má prý Tomáš Kalvoda v aktovce.}\newline
    \textit{Tomáš K.}: \uv{Ajo.}}
\section*{Sexta}
  \subsection*{Český jazyk -- prof. Martina Literáková}
    \hlod{\textit{prof}: \uv{Možná to nevíte, ale chtěla bych vám oznámit, že Honza Priessnitz vyhrál školní kolo chemické olympiády kategorie A, to znamená, že byl lepší než většina maturantů!}\newline
    \textit{Honza P.}: (\textit{důrazně}) \uv{Všichni.}}
    \hlod{(\textit{Omluvenky})\newline
    \textit{prof}: \uv{Dandi?! Ranní nevolnost? To mívají těhotné ženy!}}
    \hlod{\textit{prof}: \uv{Vláďo, který úryvek se ti líbil nejvíc?}\newline
    \textit{Vláďa}: \uv{Ten první, protože byl jediný, který jsem pochopil.}}
    \hlod{\textit{prof}: \uv{Blani, tak nám přečti svoji slohovku.}\newline
    \textit{Blanka M.}: \uv{Já jsem si vybrala jako téma trapasy. Kdybych ale věděla, že se to bude číst veřejně, tak bych si to nevybrala.}}
  \subsection*{Matematika -- prof. Pavel Boucník}
    \hlod{\textit{Petr Zelina}: \uv{Já se omlouvám, že jdu pozdě. Nemohl jsem najít klíče a doma mě zamkli. Musel jsem čekat, než mě někdo vysvobodí\ldots{}}\newline
    \textit{prof}: \uv{A v kolikátým bydlíš patře?}}
    \hlod{\textit{prof}: \uv{Honzo!? Co to máš pod lavicí?! Okamžitě to vyhoď!}\newline
    \textit{Štěpán B.}:\uv{ Né, to je moje Bravíčko!}}
    \hlod{(\textit{Jako příklad na zobrazení máme náš bodový systém + a -})\newline
    \textit{Honza P.}:\uv{Já myslím, že to je surjektivní, protože nikdo není takový tragéd, aby dostal nulu!}}
    \hlod{(\textit{Vláďovi teče krev z nosu})\newline
    Vláďa (\textit{drží si nos a utíká ze třídy}): \uv{Já musím\ldots{}}\newline
    \textit{prof}: \uv{Běžte s ním radši někdo.}\newline
    (\textit{Odchází s ním Štěpán})\newline
    \textit{Matěj Ž.}: \uv{Hej! Ale já s ním sedím!}\newline
    \textit{Honza P.}: \uv{Ale já mu dal do držky!}\newline
    \textit{prof}: \uv{Ó, vy jste se tam včera (\textit{na prodloužené}) porvali o nějakou slečnu?}}
    \hlod{\textit{prof}: \uv{Tento příklad je za tři body. Pokud ho ale budete mít špatně, strhnu vám čtyři.}}
    \hlod{\textit{prof}: \uv{Řešil jsem volitelné předměty. Všichni jste uspokojeni. Teda až na Peťku Bulantovou.}}
    \hlod{\textit{prof}: \uv{Vy se učíte tak do dějepisu!}\newline
    \textit{třída}: \uv{My musíme, že jo.}\newline
    \textit{prof}: \uv{To všechno najdete na internetu, prakticky okamžitě!}\newline
    \textit{třída}: \uv{Tak to řekněte prof. Hanákové!}\newline
    \textit{prof (\textit{zničeně})}: \uv{Já jí to říkám neustále.}}
    \hlod{\textit{prof}: \uv{Eulerovo číslo je dvě celé sedm jedna osm dva osm jedna \ldots{} Jo, umím to na dvacet desetinných míst. Když mě učil pan ředitel, tak říkal, že on sám to umí na dvacet desetinných míst. Tak jsem se to naučil taky. Ale je to úplně k ničemu.}}
    \hlod{\textit{prof}: \uv{Stačí mi, když bude mít zkoušení hlavu a patu.}\newline
    \textit{ze třídy}: \uv{Nebo prsa a zadek.}}
    \hlod{(\textit{O vodáckém kurzu})\newline
    \textit{prof}: \uv{Když si na sebe dáte neopren a pláštěnku, vytvoříte mikroklíma. Sice smrádeček, ale teplíčko.}}
  \subsection*{Matematika -- cvičení -- prof. Aleš Kobza}
    \hlod{\textit{prof}: \uv{Máte nějaké otázky?}\newline
    \textit{Zuzka K.}: \uv{Kdo byla ta slečna a proč jí tykáte?!}}
    \hlod{(\textit{Při písemce})\newline
    \textit{prof}: \uv{Dandi! Do svého!}\newline
    \textit{Dandi}: \uv{Já neopisuju, já se díval na svoje zadání. Tady!}\newline
    \textit{prof}: \uv{Dandi!}\newline
    \textit{Dandi}: \uv{Teď ale vážně.}}
    \hlod{\textit{prof}: \uv{Já jsem náhodou během svých studentských let chodil na brigády kopat hezky krumpáčem. A spoustu jsem se toho naučil. Mimo jiné ukrajinsky.}}
  \subsection*{Fyzika -- prof. Pavel Řehák }
    \hlod{\textit{prof}: \uv{Na referáty tady jsou\ldots{} zážehové motory, spalovací motory, motory u raketoplánů\ldots{}}\newline
    \textit{Zuzka K.}: \uv{Nemáte tam něco víc holčičího?}\newline
    \textit{prof}: \uv{No, je tu chladnička.}\newline
    \textit{Zuzka K.}: \uv{Tak tu beru. To je jediný slovo, kterýmu tam rozumím.}}
  \subsection*{Informatika -- prof. Viera Hájková}
    \hlod{\textit{prof}: \uv{\ldots{} opravdu, jedna z nejdůležitějších věcí je umět pochopit, co se po vás chce. Nebudu jmenovat pana zástupce, ale já často opravdu nevím, co po mně chce.}\newline
    \textit{Lucka H.}:\uv{To byste měla s námi jít do matiky někdy, abyste pochopila myšlenkové pochody profesora Boucníka!}\newline
    \textit{prof}: \uv{Ne. Pokud něco nechci v životě pochopit, pak jsou to právě myšlenkové pochody profesora Boucníka.}}
  \subsection*{Biologie -- prof. Iva Kubištová}
    \hlod{\textit{prof}: \uv{Já jsem se za svůj život chtěla aspoň třikrát stát vegetariánem. Ale když mě vždycky děsně chyběly ty řízky.}}
  \subsection*{Chemie -- prof. Veronika Kyasová}
    \hlod{\textit{prof}: \uv{Štěpáne, já vás asi přijdu něčím bacit.}\newline
    \textit{Štěpán}: \uv{A proč?}\newline
    \textit{prof}: \uv{Jen tak, abych si udělala radost a měla hezký den.}\newline
    \textit{Štěpán}: \uv{Tak fajn. Rád rozdávám radost.}}
  \subsection*{Dějepis -- prof. Alena Hanáková}
    \hlod{\textit{prof.}: \uv{\ldots{} tvrdí, že člověk už za svého života směřuje k zatracení. Když člověk krade, vraždí nebo nedává pozor v dějepisu, bude zatracen.}}
    \hlod{(\textit{Během výkladu se objevilo nějak moc Otakarů\ldots{}})\newline
    \textit{Lucka H.}: \uv{Aha! Takže ten král akorát pojmenoval svoje dítě po svém strýci\ldots{}?}\newline
    \textit{prof}: \uv{Ne! Lucie, udělejte si jasno v příbuzenských vztazích! Teda máte pravdu!}}
    \hlod{(\textit{Adam je vyvolán k tabuli a popotahuje rýmu}).\newline
    \textit{prof}: \uv{Adame, máte kapesník?}\newline
    \textit{Adam Č.}: \uv{Bohužel, právěže nemám.}\newline
    \textit{prof}: \uv{Já vám jeden tady půjčím.} (\textit{Adam smrká}) \uv{Doufám, že to umíte a nebudete pak plakat a já vám nebudu muset dát další.}}
    \hlod{\textit{prof}: \uv{Pan zástupce, kterého máte tuším z matematiky, na mě ušil takovou kulišárnu, které jste si určitě všimli na suplování. On po mně chce, abych šla suplovat k těm prckům na Příční!}\newline
    \ldots{}\ldots{}.\newline
    \textit{prof}: \uv{\ldots{} vidíte, toto je taková krásná látka, o té bychom si mohli povídat dlouho\ldots{}, pásy cudnosti\ldots{} jenomže to nám bohužel vzal profesor Boucník.}}
    \hlod{\textit{Štěpán B.}: \uv{Paní profesorko, už jsem si vybral adekvátní termín zkoušení z dějepisu.}\newline
    \textit{prof}: \uv{Ano?}\newline
    \textit{Štěpán B.}: \uv{Prosím, pátek třináctého.}}
  \subsection*{Společenské vědy -- prof. Markéta Chalupníková}
    \hlod{\textit{prof}: \uv{Po narození si musí dítě zvykat na spoustu nových věcí\ldots{} Třeba na gravitaci, která v děloze vůbec nebyla.} }
    \hlod{\textit{prof}: \uv{Zuzko, uměla jste to, učila jste se. Jedna.}\newline
    Zuzka K (\textit{cestou k lavici, mrmlá}): \uv{Neučila, ale jsou to společenský vědy, no.}}
    \hlod{\textit{prof}: \uv{V adolescenci se rozvíjí abstraktní myšlení, takže ty blbosti, co se učí v matematice\ldots{}}}
    \hlod{\textit{prof}: \uv{Pyknici. To jsou takoví ti typičtí matematici nebo informatici, hubený, pokřivený, velká hlava, ohnutý ručky\ldots{} Jé, já tu teď možná někoho urazila\ldots{}}}
  \subsection*{Výtvarná výchova -- prof. Michal Hubáček}
    \hlod{\textit{prof}: \uv{Kdo by měl zájem o exkurzi do Maunthausenu?}\newline
    \textit{Peťka B.}: \uv{Já.}\newline
    \textit{prof}: \uv{A co sestra?}\newline
    \textit{Peťka B.}: \uv{Pro tu jen jednosměrnou jízdenku.}}
\section*{Septima}
  \subsection*{Český jazyk -- prof. Martina Literáková}
    \hlod{\textit{prof}: \uv{Pošlete třídnici do matematiky, protože náš kolega iluminátor nezapsal.}}
    \hlod{\textit{prof}: \uv{Tadeáši, já ti dám zápisky! Co je tam to prasátko s vrtícím ocáskem?}}
    \hlod{\textit{Honza B.}:\uv{Ty barvy máme přiřadit k samohláskám bijektivně?}}
    \hlod{\textit{Matouš T.}:\uv{Symbolismus je velmi složitý na pochopení. Tam vlastně musí člověk  obětovat kus svého já, aby to pochopil.}\newline
    \textit{prof}: \uv{Jo, on to matematik cítí až takhle, jo?}}
    \hlod{(\textit{Krab obhajuje ZMP})\newline
    \textit{Krab}: \uv{\ldots{} on je opravdu velice starý, je mu třeba 45 let\ldots{}}\newline
    \textit{prof}: \uv{Přemýšlím, jestli si už mám jít teda kopat hrob\ldots{}?}}
    \hlod{(\textit{Dandi o surrealistickém textu})\newline
    \textit{Dandi}: \uv{Abych byl upřímný, mně to přijde jako výklad profesora Boucníka: pátý přes devátý, nechápu to\ldots{}}}
  \subsection*{Matematika -- prof. Pavel Boucník}
    \hlod{(\textit{Písemka. Lucka B. hledá pomoc všude možně.}) \newline
    \textit{Boucník}: \uv{Lucko (\textit{B.})!}\newline
    Lucka B.\uv{Tam stejně nic nemá\ldots{}}}
    \hlod{\textit{prof}: \uv{Nebudeme řešit, jestli je lepší sex v přírodě, nebo matematika!}}
    \hlod{(\textit{Na začátku června})\newline
    \textit{prof}: \uv{Tak a zítra budu zkoušet ty, co nebyli. Musíme dát nějaká znamínka!}\newline
    \textit{Lucka H.}:\uv{Pane profesore, tak mi tam rovnou napište dvě mínuska. Mně to stejně body nezmění, furt budu mít jedničku.}\newline
    \textit{prof}: \uv{Počkej, to teda ne! Co kdybych tě vyzkoušel znova?!}\newline
    \textit{Lucka H.}:\uv{Tak byste mi dal další dvě mínuska. A opět by mi to nevadilo. Takhle byste mohl klidně zkoušet 35 krát a nic by to neměnilo. Tak mě aspoň nemusíte zkoušet.}\newline
    \textit{prof}: \uv{Počkej, počkej, počkej. Tak to ne! To by znamenalo, že celý ten systém je špatný!}\newline
    Ohlušující potlesk třídy.}
  \subsection*{Matematika -- prof. Aleš Kobza}
    \hlod{\textit{prof}: \uv{Za sebou nesmí jít i-tý a j-tý velbloud.}}
    \hlod{(\textit{Čtvrtletní písemka})\newline
    \textit{prof}: \uv{Prosím vás, aspoň tu spolu přede mnou neflirtujte\ldots{}}\newline
    \textit{Tomáš K.}:\uv{To máme jako flirtovat s váma?}\newline
    \ldots{}\newline
    \textit{prof}: \uv{Anežko! Nekoukejte tam. Pojďte do první lavice. Šup!}\newline
    \textit{Anežka J.}:\uv{Proč já, a ne Matěj?}\newline
    \textit{šeptem zepředu}: \uv{Aby se mohl dívat na tebe.}\newline
    \ldots{} \newline
    \textit{Petr S.}:(\textit{mrmlá si}): \uv{\ldots{} určete pravděpodobnost\ldots{} CO?!}\newline
    \ldots{}\newline
    \textit{prof}: \uv{Za tohle bych měl dostat nějaký wellness\ldots{}}\newline
    \textit{Štěpán B.}:\uv{Na Kanáry, třeba, co?}}
    \hlod{(\textit{Lucka B. počítá týdeňák u tabule})\newline
    \textit{prof}: \uv{No tak, Lucko, dáme to spolu dohromady.}\newline
    \textit{Lucka B.(\textit{vyzývavě})}: \uv{Jóóó?}}
    \hlod{\textit{prof}: \uv{\ldots{} to je jak Nelinka. Tudle se u večeře zeptala }Maminko, a můžu říkat doprdele?\uv{}}
    \hlod{\textit{prof}: \uv{Anežko, vy tam zase datlíte do mobilu, co?}\newline
    \textit{Anežka}:\uv{Pane profesore, proč se mě na to ptáte, když je to evidentní?}\newline
    \textit{Petr Sevelda}: \uv{Tý jo, perfektní ženskej protiútok.}}
  \subsection*{Fyzika -- prof. Pavel Řehák}
    \hlod{\textit{prof}: \uv{No jó, zmizla televize.}\newline
    \textit{Tomáš K.}: \uv{Kdy byla naposledy použita?}\newline
    \textit{prof}: \uv{To bylo minulou\ldots{} válku.}}
    \hlod{(\textit{Štěpán přišel v půlce hodiny a po deseti minutách odchází}).\newline
    \textit{prof}: \uv{To byla teda krátká návštěva.}\newline
    \textit{Štěpán B.}:\uv{Já jsem sem šel více méně dobrovolně. Já sem vůbec nemusel chodit.}\newline
    \textit{prof}: \uv{A celý těch deset minut jste litoval, že jste přišel, co?}\newline
    \textit{Štěpán B.}:\uv{Jo.}}
  \subsection*{Biologie -- prof. Iva Kubištová}
    \hlod{\textit{prof}: \uv{Další formy antikoncepce?}\newline
    \textit{Lucka H}: \uv{Abstinence!}\newline
    (\textit{smích třídy})\newline
    \textit{Blanka M.}: \uv{Nám to, náhodou, s Luckou zatím vychází\ldots{}!}}
    \hlod{(\textit{Slečna posluchačka, mimochodem dcera paní profesorky třídní, zapíná dataprojektor. Musela si vylézt na lavici})\newline
    \textit{Krab}: \uv{No teda! V botech!!!}\newline
    \textit{posluchačka}: \uv{Neříkejte to mamince\ldots{}}}
  \subsection*{Dějepis -- prof. Alena Hanáková}
    \hlod{\textit{prof}: \uv{Dobrovolník k tabuli? Není? Tak, je 25. 9., čili 25. Sedlák. Ne, ten má individuál. Tak 25 - 9, to je 16, Krumlová, taky individuál! Tak 16 - 9, je 7, Buriánek.}\newline
    \textit{třída}: \uv{Ten je ve Valencii.}\newline
    \textit{prof }: \uv{Boček?}\newline
    \textit{třída}: \uv{Valencie.}\newline
    \textit{prof}: \uv{Blahynka, ten chybí. Tak nic. (\textit{začne rozpočítávat}). Ententýky, dva špalíky, čert vyletěl z elektriky, boule byla veliká, jako celá Afrika.} (\textit{David Paleček})\newline
    \textit{David Paleček}: \uv{Já už ale byl.}\newline
    \textit{prof}: \uv{Tak nic, Jirko, pojďte!}}
    \hlod{\textit{Zuzka K.}:\uv{Paní profesorko, to národní obrození budete zkoušet?}\newline
    \textit{prof}: \uv{No\ldots{}, budu, ale já ho od té doby, co jsem psala diplomovou práci na národní obrození na Moravě, tak nemám ráda!}}
    \hlod{\textit{Hanáková}: \uv{Budou Vánoce, chtělo by si to nadělit nějaký dárek.}\newline
    \textit{Peťka B.}:\uv{Můžeme Vám třeba zazpívat?}\newline
    \textit{Hanáková}: \uv{No, bez toho bych se obešla.}\newline
    \textit{Peťka B.}:\uv{Nebo naopak zpívat nebudeme?}\newline
    \textit{Hanáková}:\uv{To bych ocenila. Ale měla jsem na mysli samozřejmě písemku.}}
    \hlod{\textit{prof}: \uv{ \ldots{} ale ty obrázky neberte moc vážně. Tady vypadá Masaryk spíš jako vůdce Vietnamců.}}
  \subsection*{Společenské vědy -- prof. Chalupníková}
    \hlod{\textit{Honza P.}:\uv{No a v tom druhém kole se rozhoduje mezi dvěma kandidáty, to znamená, že aby byl jeden z nich zvolen, tak musí mít nadpoloviční většinu hlasů.}\newline
    \textit{prof}: \uv{Hm\ldots{} Já jsem to vždycky chápala tak, že stačí, když jeden kandidát má víc hlasů než ten druhý, to jsem nevěděla, že musí získat dokonce nadpoloviční většinu\ldots{}}}
    \hlod{\textit{Lucka H.}:\uv{Můžu mít dotaz? Paní profesorko, ne, že bych něco z toho plánovala, ale\ldots{} V Česku je zneužití dětí trestný, že jo. Co kdybych, ale, vyrazila někam do ciziny, kde to je povolený? Porušila bych tím něco?} \newline
    \textit{Chalupníková}: \uv{Ježiš\ldots{} no\ldots{} Nemyslím si, že taková země ale existuje\ldots{}}\newline
    \textit{Tomáš K.}:\uv{Lucko, klid, kdyby taková byla, už byste tam dávno jeli na vodu.}\newline
    \ldots{}\newline
    \textit{Lucka H}:\uv{Nebo druhej příklad, že jo. Sexuální aktivity v rámci České republiky jsou povolené od 15 let. Ale na Maltě je to 18. Je tedy možné, aby si přijeli dva šestnáctiletí Malťani užít beztrestně do Česka?}\newline
    \textit{Chalupníková}: \uv{Ale co když by pak otěhotněla?}\newline
    \textit{Lucka H}:\uv{Jenže ono těhotenství zakázaný není, tam je jenom zakázaná ta činnost, která těhotenství předchází. A tu vykonali v Česku, takže by to neměl být problém, ne?}\newline
    (\textit{Po chvíli zamyšlení}): Chalupníková: \uv{Lucko, tohle opravdu nevím. To budete muset vyzkoušet sama.}}
  \subsection*{Posluchači, suplování -- Matematika}
    \hlod{\textit{Lacinová}: "\ldots{} jo, Nezhybová ve čtvrťáku do třídnic běžně: "stacionární magnetické pole - opakování, matka moudrosti."}
  \subsection*{Posluchači, suplování -- Angličtina}
    \hlod{\textit{prof. Vomelová}: \uv{Good morning. I am Karla Vomelová and I am here today instead of Mrs. Ševčíková. I will be teaching you a bit of English today\ldots{}}\newline
    Peťka B. (\textit{nadšeně s ironií v hlase\ldots{}}):\uv{TY JO! TO JE POPRVÉ, CO TENTO ROK SLYŠÍM V ÁJINĚ ANGLIČTINU!}}
  \subsection*{Školní výlety, přestávky a jiné víceméně legální akce}
    \hlod{\textit{prof. Urban}: \uv{O facebooku, o tom mi ani nevykládejte. Nedávno jsem se na něj zaregistroval, abych zjistil, co to umí. Horší bylo, že jsem použil email mé ženy. Potom jí začaly chodit zprávy, že si mě chce přidat několik žen, které jsem v životě neviděl. Jenže vysvětlujte to doma!}}
    \hlod{\textit{prof. Urban}: \uv{Teď vám řeknu, proč byste měli chodit vepředu s učitelem (\textit{jako obvykle prof. šel o hodně před námi}).  Když jsem tu asi před 5 lety byl, tak tady dole, tam je nuda pláž. Šel jsem a byly tam nějaké dvě nahé slečny. Říkám jim, že mně to teda nevadí, ale za mnou, že jde 60 studentů. Tak mi poděkovaly a spěšně odešly. Kdyby tehdy šli studenti se mnou, viděli by ty slečny, takhle je neviděli.}}
  \subsection*{Lyžařský výcvikový kurz, Dolní Morava 2014}
    \hlod{\textit{Stupka}: \uv{Dnes jsem byl v nemocnici, zkontroloval jsem to tam. Doktoři byli super, sestřičky taky.}\newline
    \textit{Hlásenský}: \uv{Sestřicky?}\newline
    \textit{Stupka}: \uv{Jo. Tohle téma přenecháme odborníkovi, profesoru Hlásenskému.}}
  \subsection*{Výměnné pobyty}
    \hlod{\textit{Zvolská}: \uv{Prosím vás, čeká vás víkend ve vašich hostitelských rodinách. Určitě si ho užijte, hlavně jim pak poděkujte. V případě jakéhokoliv sebemenšího problému mi ihned volejte. Moje telefonní číslo máte. Takže pamatujte, hlavně opatrně\ldots{}}\newline
    \textit{Řihánková}: (\textit{přeruší prof. Zvolskou}): \uv{Prostě se neožerte!}}
    \hlod{(\textit{V lanovém centru, uvazování do popruhů})\newline
    instruktor (\textit{na prof. Nováka}): \uv{Bacha, máte to mezi nohama fakt dost utažený. Radši si to přendejte, ať vám tam něco nepochroumám na výbavičce. Nevím teda, jestli už máte splněno.}\newline
    \textit{Novák}: \uv{Mám, dvakrát. Možná kdybyste to utáhl, tak by to i leccos doma vyřešilo.}}
  \subsection*{Malino Brdo, Silvestr 2014}
    \hlod{\textit{Veselá}: \uv{To je taková španělská tradice, že si hrozny namočíme do\ldots{} uhm \ldots{} nealkoholické vodky\ldots{}}}
  \subsection*{Vodácký kurz, 2015, Moravice}
    \hlod{\textit{Boucník}: \uv{Prosím vás, teď si rozdáme vesty bijektivně. Dvě pádla budou vždy přiřazena k jedné lodi. Ježiši, to je nádhera.}}
    \hlod{(\textit{Nástup})\newline
    \textit{Boucník}: \uv{Já vám tady nic nového vlastně neřeknu.}\newline
    \textit{Štěpán B.}: \uv{Tak proč jsme sem museli jít?}}
\section*{Oktáva}
  \subsection*{Český jazyk -- prof. Martina Literáková}
    \hlod{\textit{prof}: \uv{Neexistuje, že si dáte v Praze po obědě třeba pivo.}\newline
    \textit{Honza P.}:\uv{Ááh!}\newline
    \textit{prof}: \uv{Co se děje, Honzo?}\newline
    \textit{Honza P.}:\uv{Já si jenom vzpomněl, jak je to pivo po obědě fakt dobrý.}}
  \subsection*{Matematika -- prof. Pavel Boucník }
    \hlod{\textit{Boucník}: \uv{Ale tak děcka, přece nematurujete poprvé.}}
    \hlod{\textit{Boucník}: \uv{Na to nemáme čas!}}
    \hlod{\textit{Boucník}: \uv{No, Blanko, ještě bys odmaturovala…}\newline
    \textit{Blanka M.}: \uv{To nevadí, na podzim jdu stejně kvůli chemii.}}
    \hlod{\textit{Lucka B}: \uv{To máme kontrolovat a zároveň počítat? Jsem sice žena, ale…}\newline
    \textit{Boucník}: (\textit{lišácký úsměv})}
  \subsection*{Matematika -- prof. Aleš Kobza }
    \hlod{(\textit{Tomáš Medek je drzý})\newline
    \textit{prof}: \uv{Prosím vás, kdybyste ho někdo zabil, já to na vás neřeknu.}}
  \subsection*{Německý jazyk 1 -- prof. Lucie Adámková}
    \hlod{(\textit{Štěpán něco má přeložit a vůbec netuší})\newline
    \textit{Adámková}: (\textit{záchvat smíchu}): \uv{Né, já se vám nesměju.}}
  \subsection*{Fyzika -- prof. Pavel Řehák}
    \hlod{(\textit{Ruch ve třídě})\newline
    \textit{Řehák}: \uv{Co se děje?}\newline
    \textit{Lucka H.}: \uv{Tam v tom okně je nahej chlap…}\newline
    \textit{Řehák}: (\textit{naznačí sundání trika}) \uv{Když se tady vysleču, budete mi věnovat stejnou pozornost?}}
  \subsection*{Společenské vědy -- prof. Chalupníková}
    \hlod{\textit{prof}: \uv{… a ten tvrdil, že populace roste geometrickou řadou, ale potrava pouze aritmetickou.}\newline
    \textit{Dandi}: \uv{Paní profesorko, tomuto já nerozumím. Chci požádat o vysvětlení geometrické řady.}\newline
    \textit{prof}: \uv{Ha! A to já tu náhodou mám. Takže geometrická řada je jedna, dva, čtyři, osm…}}
  \subsection*{Seminář matematických aplikací -- prof. Martin Panák}
    \hlod{(\textit{Po deseti minutách počítání a ticha ve třídě})\newline
    \textit{Vláďa Kelbl}: \uv{Jaké je zadání?}}
    \hlod{}
